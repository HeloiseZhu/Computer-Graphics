\documentclass[a4paper,UTF8]{article}
\usepackage{ctex}
\usepackage[margin=1.25in]{geometry}
\usepackage{color}
\usepackage{graphicx}
\usepackage{amssymb}
\usepackage{amsmath}
\usepackage{amsthm}
%\usepackage[thmmarks, amsmath, thref]{ntheorem}
\theoremstyle{definition}
\newtheorem*{solution}{Solution}
\newtheorem*{prove}{Proof}
\usepackage{multirow}
\usepackage{url}
\usepackage[colorlinks,urlcolor=blue]{hyperref}
\usepackage{enumerate}
\renewcommand\refname{参考文献}


%--

%--
\begin{document}
\title{\textbf{《计算机图形学》9月报告}}
\author{181860155 朱晓晴 \href{mailto:xxx@xxx.com}{heloize@126.com}}
\maketitle

\section{综述}
\dots

\section{算法介绍}
% 已完成或拟采用算法的原理介绍、自己的理解、对比分析等
\subsection{绘制线段}
\textbf{要求:}根据给定两点$(x_0,y_0)$和$(x_1,y_1)$绘制线段。

绘制线段共需完成3种算法:Naive算法、DDA算法和Bresenham算法。
其中Naive算法已提供,9月提交中完成了DDA算法和Bresenham算法。

\textbf{DDA算法}

斜率$m=\frac{y_1-y_0}{x_1-x_0}$,
若$|m|<=1$,以$x_0<x_1$为例进行说明。
以单位间隔($\Delta x=1$)对$x$进行采样,并计算对应的$y$值:

$y_{k+1}=y_k+m\;(k=0,1,...)$

若$|m|>1$,以$y_0<y_1$为例进行说明。
以单位间隔($\Delta y=1$)对$y$进行采样,并计算对应的$x$值:

$x_{k+1}=x_k+\frac{1}{m}\;(k=0,1,...)$

\textbf{Bresenham算法}

将$(x_0,y_0)$作为第一个点,
计算决策参数的第一个值$p_0=2\Delta y-\Delta x$
($\Delta x=x_1-x_0$,$\Delta y=y_1-y_0$)。

若$|m|<=1$,以$x_0<x_1$为例进行说明。


\subsection{绘制椭圆}
\dots

\subsection{其他}
		
\section{系统介绍}
% 已完成或拟采用的系统框架、交互逻辑、设计思路等
\dots

\section{总结}
\dots

\bibliographystyle{plain}%
%"xxx" should be your citing file's name.
\bibliography{xxx}



% 介绍自己系统中的巧妙的设计、额外的功能、易用的交互、优雅的代码、好看的界面等(可选)
% 注明在实现作业过程中使用的参考资料,包括技术博客等

\end{document}